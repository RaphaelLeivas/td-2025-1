%% abtex2-modelo-trabalho-academico.tex, v<VERSION> laurocesar
%% Copyright 2012-<COPYRIGHT_YEAR> by abnTeX2 group at http://www.abntex.net.br/ 

\documentclass[
	% -- opções da classe memoir --
	12pt,				% tamanho da fonte
	oneside,			% para impressão mudar para twoside para facilitar impressaõ de frente e verso 
	a4paper,			% tamanho do papel. 
	chapter=TITLE,
	sumario=tradicional,
	% -- opções do pacote babel --
	english,			% idioma adicional para hifenização
	brazil				% o último idioma é o principal do documento
]{abntex2}

% ----------------------------------------------------------
% IMPORTAÇÃO DE PACOTES
% ----------------------------------------------------------
\usepackage{helvet}
\renewcommand{\familydefault}{\sfdefault}			% Usa a fonte Arial			
\usepackage[T1]{fontenc}		% Selecao de codigos de fonte.
\usepackage[utf8]{inputenc}		% Codificacao do documento (conversão automática dos acentos)
\usepackage{indentfirst}		% Indenta o primeiro parágrafo de cada seção.
\usepackage{color}				% Controle das cores
\usepackage{graphicx}			% Inclusão de gráficos
\usepackage{microtype} 			% para melhorias de justificação
\usepackage{hyperref}
\usepackage{times}

\usepackage{amsmath}
\usepackage{amsthm,amsfonts}

\usepackage{lipsum}				% para geração de dummy text
\usepackage{import}
\usepackage{blindtext}
\usepackage{soul}
\usepackage{lscape}

\graphicspath{ {./images/} }

% ---
% Pacotes de citações
% ---
\usepackage[alf]{abntex2cite}	% Citações padrão ABNT

% ----------------------------------------------------------
% CONFIGURAÇÕES DO PDF
% ----------------------------------------------------------

% alterando o aspecto da cor azul
\definecolor{blue}{RGB}{41,5,195}
\definecolor{black}{RGB}{0,0,0}

% informações do PDF
\makeatletter
\hypersetup{
     	%pagebackref=true,
		pdftitle={\@title}, 
		pdfauthor={\@author},
    	pdfsubject={\imprimirpreambulo},
	    pdfcreator={LaTeX with abnTeX2},
		pdfkeywords={abnt}{latex}{abntex}{abntex2}{trabalho acadêmico}, 
		colorlinks=true,       		% false: boxed links; true: colored links
    	linkcolor=black,          	% color of internal links
    	citecolor=black,        		% color of links to bibliography
    	filecolor=black,      		% color of file links
		urlcolor=black,
		bookmarksdepth=4
}
\makeatother

\setlrmarginsandblock{3cm}{2cm}{*}
\setulmarginsandblock{3cm}{2cm}{*}
\checkandfixthelayout

% --- 

% ----------------------------------------------------------
% FIGURAS E TABELAS
% ----------------------------------------------------------

% ---
% Posiciona figuras e tabelas no topo da página quando adicionadas sozinhas
% em um página em branco. Ver https://github.com/abntex/abntex2/issues/170
\makeatletter
\setlength{\@fptop}{5pt} % Set distance from top of page to first float
\makeatother
% ---

% ----------------------------------------------------------
% ESPAÇAMENTOS
% ----------------------------------------------------------

% O tamanho do parágrafo é dado por:
\setlength{\parindent}{1.25cm}

% % Controle do espaçamento entre um parágrafo e outro:
\setlength{\parskip}{0.2cm} 

\setlength{\ABNTEXcitacaorecuo}{4cm}

% Espaçamento entre headers e texto abaixo
\setlength\afterchapskip{0.2cm}
\setlength\aftersecskip{0.2cm} %espaçamento entre seção e texto
\setlength\aftersubsecskip{0.2cm} %espaçamento entre subseção e texto

% ----------------------------------------------------------
% CORREÇÕES DE ESTILO
% ----------------------------------------------------------

% Estilos das legendos
\captionnamefont{\ABNTEXfontereduzida}
\captiontitlefont{\ABNTEXfontereduzida}
\setlength{\belowcaptionskip}{1pt} % espaçamento depois do título das tabelas/figuras
\setlength{\abovecaptionskip}{1pt} % espaçamento antes da legenda de tabelas/figuras

% Estilos dos títulos

\renewcommand{\ABNTEXchapterfontsize}{\bfseries\normalsize}
\renewcommand{\ABNTEXsectionfontsize}{\itshape\normalsize}
\renewcommand{\ABNTEXsubsectionfontsize}{\normalfont\normalsize}
\renewcommand{\ABNTEXsubsubsectionfontsize}{\normalfont\normalsize}

\renewcommand{\chaptitlefont}{\normalfont\bfseries}
\setsecheadstyle{\normalfont\itshape}
\setsubsecheadstyle{\normalfont}
\setsubsubsecheadstyle{\normalfont}

\renewcommand{\ABNTEXchapterfont}{\bfseries}
\renewcommand{\ABNTEXchapterfontsize}{\normalsize}
\setboolean{ABNTEXupperchapter}{true}

% Estilos nos sumários
\renewcommand{\cftchapterfont}{\MakeUppercase}
\setboolean{ABNTEXupperchapter}{true}

\renewcommand{\cftsectionfont}{\normalfont}
\renewcommand{\cftsubsectionfont}{\normalfont}
\renewcommand{\cftsubsubsectionfont}{\normalfont}

% ----------------------------------------------------------
% COMPILA O ÍNDICE
% ----------------------------------------------------------
\makeindex

% ----------------------------------------------------------
% COMANDOS CUSTOMIZADOS
% ----------------------------------------------------------
\newcommand{\un}[1]{\;\text{#1}}
\newcommand{\logo}{\quad \Rightarrow \quad}
\newcommand{\codeword}[1]{\texttt{\textcolor{black}{#1}}}
\newcommand{\specialcell}[2][c]{%
  \begin{tabular}[#1]{@{}c@{}}#2\end{tabular}}

% ----------------------------------------------------------
% INÍCIO DO DOCUMENTO
% ----------------------------------------------------------
\begin{document}

%\selectlanguage{english}
\selectlanguage{brazil}

% Retira espaço extra obsoleto entre as frases.
\frenchspacing 

% ----------------------------------------------------------
% ELEMENTOS PRÉ-TEXTUAIS
% ----------------------------------------------------------
\import{./elementos-pre-textuais}{capa.tex}
\import{./elementos-pre-textuais}{sumario.tex}

% % ----------------------------------------------------------
% % ELEMENTOS TEXTUAIS
% % ----------------------------------------------------------
\textual

\pagestyle{simple}
	
\chapter{Introdução}\label{cap:introducao} 

\lipsum[1-2]

\chapter{Modelagem}\label{cap:modelagem} 

Temos que modelar dois problemas mono-objetivos:

\begin{itemize}
	\item Problema 1: minimização do custo de manutenção total $f_1 (\cdot)$
	\item Problema 2: minimização do custo esperado de falha total $f_2 (\cdot)$
\end{itemize}

\section{Problema 1}

Temos essencialmente um problema de designação simples. Seja $N$ o número de 
equipamentos e $J$ o número de políticas de manutenção, definimos a variável de decisão $x_{ij}$ 
por

\begin{equation}
	x_{ij}: \un{se a máquina $i$ executa a manutenção $j$}
\end{equation}

\noindent onde 

\[ x_{ij} \in \{0,1\} \quad , \quad i = \{1, 2, ..., N\}  \quad , \quad j = \{1, 2, ..., J\} \]

Para a função objetivo, seja $c_j$ o custo de executar a manutenção $j$. Note que esse custo 
independe da máquina $i$ que estamos executando a manutenção. Temos a função objetivo  

\begin{equation}\label{eq:objf1}
	\min f_1 = \sum_{i=1}^{N} \sum_{j=1}^{J} c_j x_{ij}
\end{equation}

\noindent sujeito a 

\begin{equation}\label{eq:restf1}
	\sum_{j=1}^{J} x_{ij} = 1 \quad , \quad \forall i = {1, 2, ..., N}
\end{equation}

A \autoref{eq:restf1} indica que toda máquina executa exatamente uma política de manutenção.
Além disso, note que solução de \autoref{eq:restf1} é trivial: basta escolher o plano de 
manutenção com o menor custo para todos os equipamentos.

\section{Problema 2}

O custo da falha de cada equipamento é dado pelo produto da probabilidade de falha $p_{ij}$ pelo custo da falha do equipamento,
dada por $d_i$. Assim, temos 

\begin{equation}\label{eq:objf2}
	\min f_2 = \sum_{i=1}^{N} \sum_{j=1}^{J} p_{ij} \, d_i \, x_{ij}
\end{equation}

\noindent onde 

\[ x_{ij} \in \{0,1\} \quad , \quad i = \{1, 2, ..., N\}  \quad , \quad j = \{1, 2, ..., J\} \]

\[ 
p_{ij} = \frac{F_i \left(t_0 + k_j \Delta t \right) - F_i\left(t_0\right) }{1 - F_i\left(t_0\right)}
\quad , \quad 
F_i(t) = 1 - \exp \left[ - \left( \frac{t}{\eta_i} \right)^{\beta_i} \right] 
\]

\noindent sujeito a 

\begin{equation}\label{eq:restf2}
	\sum_{j=1}^{J} x_{ij} = 1 \quad , \quad \forall i = {1, 2, ..., N}
\end{equation}

Note que na \autoref{eq:objf2} temos essencialmente um problema de programação linear inteira.
Em \autoref{eq:objf2} temos essencialmente um problema de programação linear inteira.

\section{Modelagem Multiobjetivo}

\chapter{Algoritmos}\label{cap:algoritmos} 

\lipsum[1-2]


\chapter{Resultados e Discussão}\label{cap:resultados} 

\lipsum[1-2]


\chapter{Conclusão}\label{cap:conclusao} 

\lipsum[1-2]


\chapter{Referências}\label{cap:referencias} 

\lipsum[1-2]


\end{document}
